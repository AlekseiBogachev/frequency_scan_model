\section{Заключение}

Применение многоэкспоненциальной модели, как показано в таблицах 
\ref{table:model_comparison_263}, \ref{table:model_comparison_283},
\ref{table:model_comparison_303} позволяет повысить точность моделирования
частотных сканов. При этом, важно помнить, что моногоэкспоненциальная модель
никак не учитывает искажения частотного скана за счёт влияния нелинейностей
измерительного тракта.

Инициализация моногоэкспоненциальной модели случайными начальными параметрами
может влиять на результат решения и приводить к тому, что алгоритм будет
останавливаться вблизи оптимальных значений параметров, но оптимальных достигать
не будет (рисунок \ref{pic:multi_exp_model_283}), в таких случаях необходимо
выполнять идентификацию повторно, либо подстраивать параметры алгоритма 
идентификации, например скорость градиентного спуска.

Многоэкспоненциальная модель нуждается в доработке алгоритма идентификации,
в частности, оптимизации с точки зрения времени расчётов.

На графиках в координатах Аррениуса наблюдается уширение <<корридора>> 
образованного линиями регресси, как для линейных регрессий, так и для 
регрессий полиномом второй степени (рисунки 
\ref{pic:arrhenius_lin_regr_with_inset} и 
\ref{pic:arrhenius_poly_regr_with_inset}). Для анализа результатов в 
коориднатах Аррениуса и вычисления энергии активации необходимо собрать больше 
данных: получить больше частотных при различных температурах.