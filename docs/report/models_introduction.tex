\section{Введение}
Релаксационная спектроскопия глубоких уровней (РСГУ) -- метод исследования 
электрически активных дефектов в полупроводниковых материалах. Данный метод 
обладает высокой чувствительностью к малым концентрациям ловушек носителей 
зарядов и является спектроскопическим. Существуют различные вариации РСГУ, 
например токовая и емкостная, так метод емкостной релаксационной спектроскопии 
глубоких уровней основан на исследовании процесса (сигнала) релаксации ёмкости 
барьерной структуры. Определив значения постоянной времени сигнала релаксации 
для разных температур образца, исследователь может определить энергию активации 
дефекта, вызывающего релаксацию. 

В идеальном случае релаксация ёмкости носит экспоненциальный характер, однако 
так бывает не всегда. Согласно обзору \cite{istratov_exp_analysis} 
сигналы релаксации ёмкости барьерных структур можно условно разделить на три 
группы:
    \begin{enumerate}
        \item Моноэкспоненциальный сигнал релаксации, обусловленный одним 
        единственным энергетическим уровнем в запрещённой зоне полупроводника.
        \item Сигнал релаксации, состоящий из суммы нескольких 
        моноэкспоненциальных сигналов релаксации.
        \item Сигнал релаксации, характеризуемый непрерывным распределением 
        скоростей эмиссии, представленным спектральной функцией $g(\lambda)$.
    \end{enumerate}
В последних двух случаях сигнал не является экспоненциальным. 

Задача определения постоянной времени одной или нескольких экспоненциальных 
составляющих процесса релаксации ёмкости или спектральной функции $g(\lambda)$ 
отностися к задачам экспоненциального анализа. Подходы к решению таких задач,
а также фундаментальные ограничения и особенности сбора и обработки 
экспериментальных данных рассмотрены в исчерпывающем обзоре 
\cite{istratov_exp_analysis}.

Во Владимирском Государственном университете создан измерительно-вычислительный
комплекс релаксационной спектроскопии глубоких уровней, основным измерительным
прибором которого служит спектрометр DLS"~82E фирмы Semilab. 
Измерительно-вычислительный комплекс реализует метод емкостной РСГУ с частотным
сканированием при постоянной температуре. В спектрометре аппаратно реализована
корреляционная обработка сигнала релаксации с опорной функцией lock-in. 

Технические решения, заложенные в названном измерительном оборудовании, требуют
особых методов анализа полученных экспериментальных данных, в частности,
разработки и идентификации моделей частотных сканов (экспериментальных данных, 
полученных на измерительно-вычислительном комплексе).

В следующих разделах отчёта будут рассмотрены модели частотных сканов, некоторые
технические и методические вопросы их реализации и идентификации, а также их
аппробация на экспериментальных данных.

В современной учебной и технической литературе набирает популярность термины 
<<машинное обучение>> и <<статистическое обучение>> (например в книгах 
\cite{hands_on_ml}, \cite{nikolenko_deep_learning}, 
\cite{elements_of_statistical_learning}). За ними, как правило, скрывается
процесс создания моделей, которые после идентификации их параметров на некой
тренировачной выборке (экспериментальных данных с известными целевыми 
значениями), способны с некоторой точностью определять целевые значения для
новых, ранее невстречавшихся данных. Таким образом, регрессия и классификация
являются одними из самых частых задач машинного обучения \cite{hands_on_ml}.
В данном отчёте термин <<машинное обучение>> будет использоваться именно для 
обозначения процесса разработки и идентификации моделей, решения задачи 
регрессии.