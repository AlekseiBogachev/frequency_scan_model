\chapter{Цели и задачи}

Согласно обзору \cite{istratov_exp_analysis} сигналы релаксации ёмкости 
барьерных структур можно условно разделить на три группы:
\begin{enumerate}
    \item Моноэкспоненциальный сигнал релаксации, обусловленный одним  
    энергетическим уровнем в запрещённой зоне полупроводника.
    \item Сигнал релаксации, состоящий из суммы нескольких моноэкспоненциальных 
    сигналов релаксации.
    \item Сигнал релаксации, характеризуемый непрерывным распределением скоростей 
    эмиссии, представленным спектральной функцией $g(\lambda)$.
\end{enumerate}

Таким образом, цель работы по моделированию частотных сканов можно описать 
следующим образом: найти способ определять спектральную функцию 
$g(\lambda)$, её параметры, а также зависимость этих параметров от температуры.

Для достижения поставленной цели нужно решить следующие задачи:
\begin{enumerate}
    \item Разработать алгоритм идентификации частотного скана 
    моноэкспоненциального сигнала релаксации.
    \item Разработать алгоритм идентификации частотного скана 
    неэкспоненциального сигнала релаксации с показателем $p$, 
    характеризующим нелинейность и неэкспоненциальность.
    \item Разработать программу идентификации частотного 
    скана сигнала релаксации, состоящего из суммы моноэкспоненциальных 
    сигналов. Программа должна определять параметры каждого 
    моноэкспоненциального сигнала.
    \item Разработать программу идентификации группы частотных сканов 
    при разных температурах. На данном этапе предполагается, что сигнал 
    релаксации образован суммой моноэкспоненциальных сигналов, 
    количество которых не меняется в зависимости от температуры, 
    меняются только их параметры.
    \item Разработать программу иденитификации спектральной 
    функции $g(\lambda)$ и её параметров для отдельного 
    частотного скана.
    \item Разработать программу иденитификации спектральной 
    функции $g(\lambda)$ и её параметров для группы частотных сканов при 
    разных температурах, полагая, что от температуры зависят только 
    параметры $g(\lambda)$.
\end{enumerate}