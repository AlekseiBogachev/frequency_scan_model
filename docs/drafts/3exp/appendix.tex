\section*{Приложение}
\addcontentsline{toc}{section}{Приложение 1}
	\begin{table}[ht]
	\begin{tabular}{|l|l|l|l|l|l|l|l|l|l|l|}
        \hline
		№  & $A$  & $\log10(\tau)$ & $p$   & $E$      & $\tau_0$ & $\tau_1$ & $\tau_2$ & $A_0, A_1, A_2$\\ \hline
		0  & 3    & -2             & 1     & 7.65E-10 & 0.01     & 0.01     & 0.01     & 1              \\ \hline
		1  & 3    & -2             & 0.999 & 3.04E-08 & 0.0104   & 0.01     & 0.00962  & 1              \\ \hline
		2  & 3    & -2             & 0.997 & 4.73E-07 & 0.0108   & 0.01     & 0.00926  & 1              \\ \hline
		3  & 2.99 & -2             & 0.994 & 2.38E-06 & 0.0112   & 0.01     & 0.00891  & 1              \\ \hline
		4  & 2.98 & -2             & 0.989 & 7.47E-06 & 0.0117   & 0.01     & 0.00858  & 1              \\ \hline
		5  & 2.97 & -2             & 0.984 & 1.81E-05 & 0.0121   & 0.01     & 0.00825  & 1              \\ \hline
		6  & 2.96 & -2             & 0.976 & 3.71E-05 & 0.0126   & 0.01     & 0.00794  & 1              \\ \hline
		7  & 2.95 & -2             & 0.968 & 6.78E-05 & 0.0131   & 0.01     & 0.00764  & 1              \\ \hline
		8  & 2.93 & -2             & 0.959 & 0.000114 & 0.0136   & 0.01     & 0.00736  & 1              \\ \hline
		9  & 2.92 & -2             & 0.948 & 0.00018  & 0.0141   & 0.01     & 0.00708  & 1              \\ \hline
		10 & 2.9  & -2             & 0.937 & 0.000269 & 0.0147   & 0.01     & 0.00681  & 1              \\ \hline
		11 & 2.88 & -2             & 0.924 & 0.000386 & 0.0153   & 0.01     & 0.00656  & 1              \\ \hline
		12 & 2.85 & -2             & 0.911 & 0.000535 & 0.0158   & 0.01     & 0.00631  & 1              \\ \hline
		13 & 2.83 & -2             & 0.897 & 0.000719 & 0.0165   & 0.01     & 0.00607  & 1              \\ \hline
		14 & 2.81 & -2             & 0.882 & 0.000943 & 0.0171   & 0.01     & 0.00584  & 1              \\ \hline
		15 & 2.78 & -2             & 0.866 & 0.00121  & 0.0178   & 0.01     & 0.00562  & 1              \\ \hline
		16 & 2.75 & -2             & 0.85  & 0.00152  & 0.0185   & 0.01     & 0.00541  & 1              \\ \hline
		17 & 2.72 & -2             & 0.834 & 0.00188  & 0.0192   & 0.01     & 0.00521  & 1              \\ \hline
		18 & 2.69 & -2             & 0.817 & 0.00229  & 0.02     & 0.01     & 0.00501  & 1              \\ \hline
		19 & 2.66 & -2             & 0.799 & 0.00274  & 0.0207   & 0.01     & 0.00482  & 1              \\ \hline
		20 & 2.63 & -2             & 0.781 & 0.00325  & 0.0215   & 0.01     & 0.00464  & 1              \\ \hline
		21 & 2.6  & -2             & 0.763 & 0.00381  & 0.0224   & 0.01     & 0.00447  & 1              \\ \hline
		22 & 2.57 & -2             & 0.745 & 0.00441  & 0.0233   & 0.01     & 0.0043   & 1              \\ \hline
		23 & 2.53 & -1.99          & 0.727 & 0.00506  & 0.0242   & 0.01     & 0.00414  & 1              \\ \hline
		24 & 2.5  & -1.99          & 0.708 & 0.00574  & 0.0251   & 0.01     & 0.00398  & 1              \\ \hline
		25 & 2.47 & -1.99          & 0.69  & 0.00647  & 0.0261   & 0.01     & 0.00383  & 1              \\ \hline
		26 & 2.43 & -1.99          & 0.671 & 0.00723  & 0.0271   & 0.01     & 0.00369  & 1              \\ \hline
		27 & 2.4  & -1.99          & 0.653 & 0.00803  & 0.0282   & 0.01     & 0.00355  & 1              \\ \hline
		28 & 2.36 & -1.99          & 0.635 & 0.00884  & 0.0293   & 0.01     & 0.00341  & 1              \\ \hline
		29 & 2.33 & -1.99          & 0.616 & 0.00968  & 0.0304   & 0.01     & 0.00329  & 1              \\ \hline
		30 & 2.29 & -1.98          & 0.598 & 0.0105   & 0.0316   & 0.01     & 0.00316  & 1              \\ \hline
	\end{tabular}
\end{table}
    В таблице \ref{table:results} использованы следующие обозначения:
    \begin{description}
    	\item[$\tau_0, \tau_1, \tau_2$] -- Заданные постоянные времени
    	экспоненциальных составляющих на спектре в секундах.
    	\item[$A_0, A_1, A_2$] -- Заданные амплитуды экспоненциальных 
        составляющих на спектре в условных единицах. В данном случае 
        амплитуды всех составляющих равны 1.
    	\item[$p$] -- коэффициент нелинейности"=неэкспоненциальности,
    	полученный в результате идентификации параметров модели. 
        Безразмерная величина.
    	\item[$\tau$] -- постоянная времени, полученная в результате
    	идентификации параметров модели, в секундах.
    	\item[$A$] -- амплитуда, полученная в результате идентификации
    	параметров модели, в условных единицах.
    	\item[$E$] -- среднеквадратическая ошибка (квадрат единиц 
        амплитуды $A$).
    \end{description}