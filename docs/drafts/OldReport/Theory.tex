\documentclass{report}

\usepackage[utf8]{inputenc}
\usepackage[T1, T2A]{fontenc}
\usepackage[english, russian]{babel}
\usepackage{csquotes}

\usepackage[
    backend = biber,
    style = numeric,
]{biblatex}

\addbibresource{Refs.bib}

\usepackage{graphicx}
\graphicspath{ {./images/} }

\title{Реферат статей по РСГУ}
\author{Богачев А.М.}
\date{\today}

\begin{document}
    \maketitle

    \chapter*{Список сокращений}
    \begin{description}
        \item[ICTS] -- Isotermal Capacitance Transient Spectroscopy;
        \item[ГУ] -- Глубокий Уровень;
        \item[РСГУ] -- Релаксационная Спектроскопия Глубоких Уровней;
    \end{description}


    \chapter{Методы измерений в РСГУ}

    \section{Isotermal Capacitance Transient Spectroscopy (ICTS)}
    \subsection{Введение}
    В статье \cite{Okushi_1981} авторы описывают предложенный ими метод 
    измерения параметров глубоких уровней в полупроводниках -- ICTS,
    Isotermal Capacitance Transient Spectroscopy, что почти дословно
    можно перевести как <<Емкостная релаксационная спектроскопия при 
    постоянной температуре>>. Особенность данного метода в том, что
    измерения длительности переходного процесса происходят в условиях
    постоянной температуры. 
    
    Метод является разновидностью емкостной релаксационной 
    спектроскопии глубоких уровней. Корреляционная обработка не 
    используется.

    
    
    \subsection{Описание метода}
    При постоянном напряжении обратного смещения $V_R$ ёмкость области
    объёмного заряда (depletion layer) $C_0$ p\textsuperscript{+}n-перехода
    определяется формулой...






    \printbibliography

\end{document}