\chapter{Физическая модель}

    В данном разделе представленно описание модели частотного скана 
    в физических формулах.

    Согласно публикации \cite{LangDLTS} скорость эмиссии носителей
    заряда определяется выражением \ref{eq:emission_rate}. В
    выражении \ref{eq:emission_rate} приведены обозначения для
    неосновных носителей. Аналогичное выражение может быть написано 
    для основных.
    \begin{equation}
        \label{eq:emission_rate}
        e_1 = \left(\sigma_1 \langle\upsilon_1\rangle N_{D1}/g_1\right)\exp
        \left(\frac{-\Delta E}{kT}\right),
    \end{equation}
    где
    \begin{description}
        \item[$\sigma_1$] -- сечение захвата носителей заряда,
        \item[$\langle\upsilon_1\rangle$] -- средняя тепловая скорость
        носителей заряда,
        \item[$N_{D1}$] -- эффективная плотность состояний в зоне неосновных
        носителей заряда,
        \item[$g_1$] -- фактор вырождения уровня соответствующего глубокого
        уровня,
        \item[$\Delta E$]  -- энергия активации глубокого уровня,
        \item[$k$] -- постоянная Больцмана,
        \item[$T$] -- абсолютная температура.
    \end{description}

    Стандарт ASTM F978-02 предлагает ввести предэкспоненциальный множитель
    $B$, определяемый выражением \ref{eq:pre_exp_factor}, и вычислять его
    при анализе результатов измерений.
    \begin{equation}
        \label{eq:pre_exp_factor}
        B = \frac{\sigma_1 \langle\upsilon_1\rangle N_{D1}}{g_1 T^2}
    \end{equation}

    С учётом предэкспоененциального множителя $B$ выражение 
    \ref{eq:emission_rate} принимает вид выражения \ref{eq:emission_rate_ASTM}.
    \begin{equation}
        \label{eq:emission_rate_ASTM}
        e_1 = BT^2\exp\left(\frac{-\Delta E}{kT}\right).
    \end{equation}
    
    Соотвтетственно, постоянная времени релаксации ёмкости, обратнопропорциональная
    скорости эмиссии носителей заряда, определяется выражением \ref{eq:tau}.
    \begin{equation}
        \label{eq:tau}
        \tau_1 = \frac{1}{e_1} = \frac{\exp\left(\frac{\Delta E}{kT}\right)}{BT^2}.
    \end{equation}

    Согласно обзорам \cite{Peaker_DLTS_review_2018} и \cite{Tin_DLTS_2012} сигнал 
    релаксации ёмкости p-n-перехода, вызванный одним глубоким уровнем носит
    экспоненциальный характер и может быть описан выражением
    \ref{eq:cap_decay_maj}, если он вызван влиянием ловушек основных носителей
    зряда, и выражением \ref{eq:cap_decay_min}, если он вызван ловушками 
    неосновных ностиелей заряда.
    \begin{equation}
        \label{eq:cap_decay_maj}
       \Delta C\left(t\right) = C_0-A\exp{\left(\frac{-t}{\tau}\right)},
    \end{equation}
    \begin{equation}
        \label{eq:cap_decay_min}
        \Delta C\left(t\right) = C_0+A\exp{\left(\frac{-t}{\tau}\right)},
    \end{equation}
    где 
    \begin{enumerate}
        \item[$C_0$] -- барьерная ёмкость p-n-перехода в начальный момент времение,
        \item[$A$] -- амплитуда сигнала релаксации ёмкости.
    \end{enumerate}

    В спектрометре DLS-82E используется корреляционный метод обработки сигнала
    релаксации ёмкости, при этом перед корреляционной обработкой из сигнала
    вычитается постоянная составляющая. Таким образом, выражения 
    \ref{eq:cap_decay_maj}, \ref{eq:cap_decay_min} принимают следующий вид:
    \begin{equation}
        \label{eq:cap_decay}
        \Delta C\left(t\right) = A\exp{\left(\frac{-t}{\tau}\right)},
    \end{equation}


    Выходной сигнал коррелятора спектрометра DLS-82E определяется 
    выражением \ref{eq:eq1}.

    \begin{equation}
        \label{eq:eq1}
        S\left(\tau,C_A,F_0, t_1\right) = C_A K_{BS} K_{LS} \phi\left(\tau,F_0,t_1\right),
    \end{equation}
    где
    \begin{description}
        \item[$C_A$] -- амплитуда емкостного релаксационного сигнала,
        \item[$K_{BS}$] -- масштабный коэффициент, зависящий от чувствительности 
        емкостного моста,
        \item[$K_{LS}$] -- масштабный коэффициент селектора,
        \item[$\tau$] -- постоянная времени релаксации гулбокого уровня,
        \item[$F_0$] -- частота сканирования импульсов заполнения,
        \item[$t_1$] -- длительность импульса заполнения,
        \item[$\phi\left(\tau,F_0,t_1\right)$] -- функция определяемая выражением
        \ref{eq:eq2}.
    \end{description}
    \begin{equation}
        \label{eq:eq2}
        \phi\left(\tau,F_0,t_1\right) = 
        M \tau F_0 e^{-\frac{0.05}{\tau F_0}}
        \left(1-e^{\frac{t_1 F_0-0.45}{\tau F_0}}
        -e^{-\frac{0.5}{\tau F_0}}+
        e^{\frac{t_1 F_0-0.95}{\tau F_0}}\right),
    \end{equation}
    где $M$ -- масштабный множитель, определяемый выражением \ref{eq:scale_factor}.
    \begin{equation}
        \label{eq:scale_factor}
        M = \frac{1}{\max{\left[
        \tau F_0 e^{-\frac{0.05}{\tau F_0}}
        \left(1-e^{\frac{t_1 F_0-0.45}{\tau F_0}}
        -e^{-\frac{0.5}{\tau F_0}}+
        e^{\frac{t_1 F_0-0.95}{\tau F_0}}\right)
        \right]}}
    \end{equation}