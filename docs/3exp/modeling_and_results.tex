\section{Моделирование и его результаты}
	

	\subsection{Реализация модели}
	Модель (выражение \ref{eq:dls82e_model_S_p}) реализована на 
	языке программирования Python (версия 3.9.12) с применением 
	библиотеки TensorFlow (версия 2.8.0) и других библиотек	для научных
	вычислений.

	Модель частотного скана реализованна в виде отдельного класса \\
	\emph{FrequencyScan()} в модуле \emph{fsmodels.py}. Код 
	прокоментирован. Все параметры снабжены адекватными значениями по 
	умолчанию. В дальнейшем планируется дополнение документации и 
	перенос всех программных инструментов для обработки 
	экспериментальных данных в один пакет.

	Модель реализует две функции:
	\begin{enumerate}
		\item Вычисление частотного скана по заданным параметрам и 
		заданному вектору десятичных логарифмов частот опорной функции.
		\item Идентификация параметров модели частотного скана по 
		экспериментальным данным.
	\end{enumerate}
	Имеется возможность вывода значений параметров модели на каждой 
	итерации при идентификации. Примеры использования модели можно найти
	в файле \emph{tensorflow\_model.ipynb} (ПО Jupyter Notebook в составе 
	дистрибудтива Anaconda).

	Программа при каждом вычислении значения $\phi\left(\tau,F_0,
	t_1\right)$	(выражение \ref{eq:dls82e_model_phi}) находит 
	\(
		\max{\left[
	    \tau F_0 e^{-\frac{0.05}{\tau F_0}}
	    \left(1-e^{\frac{t_1 F_0-0.45}{\tau F_0}}
	    -e^{-\frac{0.5}{\tau F_0}}+
	    e^{\frac{t_1 F_0-0.95}{\tau F_0}}\right)
	    \right]}
    \)
	методом	градиентного спуска и вычисляет масштабный множитель $M$ 
	(выражение \ref{eq:dls82e_model_M}).

	Идентификация параметров модели производится методом градиетного 
	спуска, при этом минимизируется среднеквадратическая ошибка между 
	значениями, полученными в результате измерений, и результатами 
	моделирования (выражение \ref{eq:mse}).
	\begin{equation}
		\label{eq:mse}
		E = \frac{1}{n}\sum_{i=1}^{n}\left(y_i - y_i^*\right)^2,
	\end{equation}
	где
	\begin{description}
		\item[$y_i$] -- значения, полученные в результате измерений,
		\item[$y_i^*$] -- значения, полученные в результате моделирования,
		\item[$n$] -- количество измерений.
	\end{description}

	Градиентный спуск везде реализован с помощью библиотеки TensorFlow,
	которая использует алгоритм дифференцирования на графе вычислений,
	таким образом, производная берётся символьно (точно), затем вычисляется
	её значение, по этому точность вычисления градиента ограничена только
	разрядностью чисел \cite{hands_on_ml}.

	\textbf{Для ускорения процесса идентификации и улучшения сходимости
	в модели вместо постоянной времени сигнала релаксации $\tau$ 
	выполняется идентификация величины $\rho = log10(\tau)$. По этим же
	и некоторым	другим техническим причинам при вычислении частотного 
	скана на вход модели нужно подавать не вектор частот опорной 
	функции, а вектор их десятичных логарифмов.}

	На рисунке \ref{pic:identification_test} показан пример результата
	идентификации модели на тестовых (специально сгенерированных) данных.

	\begin{figure}[ht]
		\centering
		\includegraphics[width=0.75\textwidth]{identification_test}
		\caption{Пример результата идентификации модели.}
		\label{pic:identification_test}
	\end{figure}

	На рисунке \ref{pic:param_path} показан <<путь>> изменеия параметров 
	(постоянной времени и амплитуды) при идентификации. Красными точками 
	отмечены значения параметров на каждой итерации, изолинии показывают 
	значения среднеквадратической ошибки.

	\begin{figure}[ht]
		\centering
		\includegraphics[width=\textwidth]{path}
		\caption{<<Путь>> изменеия параметров при идентификации.}
		\label{pic:param_path}
	\end{figure}


	\subsection{Результаты моделирования в отсутствии шума}
	\begin{figure}[ht]
		\centering
		\begin{subfigure}{\textwidth}
			\includegraphics[width=0.9\linewidth]{identification_results_0}
			\caption{Рассчитанный частотный скан и результаты идентификации}
			\label{pic:results_0_scan}
		\end{subfigure}
		\begin{subfigure}{0.75\textwidth}
			\includegraphics[width=0.9\linewidth]{spectr0}
			\caption{Спектр}
			\label{pic:results_0_spectr0}
		\end{subfigure}
		\caption{Пример №0}
		\label{pic:result_0}
	\end{figure}