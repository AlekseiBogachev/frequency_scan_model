\section{Выводы}
	\begin{enumerate}
		\item Полученные результаты моделирования частотных сканов 
		(рисунок~\ref{pic:p_delta_tau}, таблица~\ref{table:results})
		позволяют сделать вывод о том, что коэффициент 
		нелинейности"=неэкспоненциальности $p$ нелинейно убывает с 
		увеличением разности постоянных времени крайних линий на спектре
		($\tau_0 - \tau_2$).
		\item С увеличение разности постоянных времени крайних линий на 
		спектре растёт среднеквадратическая ошибка между результатами,
		полученными на идентифицированной модели, и исходными данными.
		\item Возможно улучшение программной реализации модели и повышение
		скорости идентификации её параметров.
	\end{enumerate}