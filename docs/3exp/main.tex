\documentclass[14pt]{extarticle}

\usepackage{geometry}
\geometry{a4paper,
          total={165mm,257mm},
          left=30mm,
          top=20mm,
         }

\usepackage[utf8]{inputenc}
\usepackage[T1, T2A]{fontenc}
\usepackage[english, russian]{babel}
\usepackage{csquotes}

\usepackage[
    backend = biber,
    style = numeric,
]{biblatex}

\addbibresource{Refs.bib}

\usepackage{graphicx}
\graphicspath{ {./images/} }

\usepackage{subcaption}

% \usepackage{amsmath}


\title{Моделирование частотных сканов}
\author{Богачев А.М.}
\date{\today}


\begin{document}

    \maketitle
    \begin{abstract}
        В отчёте приведены результаты моделирования частотных сканов,
        образованных сигналом релаксации, состоящим из трёх 
        экспоненциальных сигналов с единичной амплитудой. При моделировании
        не учитывается влияние шумов. Показано, что показатель 
        нелинейности"=неэкспоненциальности $p$ убывает с увеличением 
        разницы между постоянными времени крайних состоавляющих на спектре, 
        при этом растёт среднеквадратическая ошибка между результатами 
        измерений и результатами моделирования. В отчёте также приведена 
        математическая модель частотного скана и краткое описание её 
        программной реализации, в графическом виде преведены примеры 
        результатов идентификации модели, приведён график зависимости $p$ 
        от разности между постоянными времени крайних составляющих на 
        спектре, в приложении приведена таблица с результатами 
        моделирования частотных сканов.
    \end{abstract}
    \tableofcontents

    \pagebreak

    \section{Цели и задачи}

\emph{Цель работы}: изучить влияние расположения линий на спектрах на 
коэффициент нелинейности"=неэкспоненциальности $p$.

Для достижения поставленной цели нужно решить следующие задачи:
\begin{enumerate}
    \item Разработать программу идентификации частотного скана. Модель
    частотного скана должна учитывать коэффициент 
    нелинейности"=неэкспоненциальности $p$.
    \item Расчитать частотные сканы для разных спектров.
    \item Выполнить идентификацию полученных сканов.
    \item Построить зависимость коэффициента $p$ от расстояния между 
    крайними линиями на спектре.
\end{enumerate}

    \section{Математическе модели}

    В данном разделе представленно описание модели частотного скана 
    в математических выражениях.


    \subsection{Модель сигнала релаксации ёмкости}

    Согласно обзору \cite{istratov_exp_analysis}, зависимость значения
    ёмкости от времени $f(t)$ для моноэкспоненциального сигнала релаксации
    имеет вид выражения \ref{eq:monoexp}.
    \begin{equation}
        \label{eq:monoexp}
        f(t) = A \exp \left(-\lambda t\right) ,
    \end{equation}
    где
    \begin{description}
        \item[\(A\)] -- амплитуда сигнала релаксации ёмкости;
        \item[\(\lambda\)] -- скорость экспоненциального спада,
        обратнопрпорциональная постоянной веремени сигнала релаксации
        $\tau$ (выражение \ref{eq:lambda}).
    \end{description}
    \begin{equation}
        \label{eq:lambda}
        \lambda = \tau ^ {-1}
    \end{equation}
    Спектр моноэкспоненциального сигнала релаксации имеет вид, 
    представленный на рисунке \ref{pic:monoexp_spect_example}.
    \begin{figure}[h!]
        \centering
        \includegraphics[width=0.5\textwidth]{monoexp_spect_expample}
        \caption{Пример спектра моноэкспоненциального сигнала релаксации
        ёмкости.}
        \label{pic:monoexp_spect_example}
    \end{figure}

    Согласно источнику \cite{istratov_exp_analysis}, зависимость сигнала 
    релаксации ёмкости от времени $f(t)$ для сгинала, образованного 
    несколькими дискретными экспоненциальными сигналами, определяется 
    выражением 
    \ref{eq:discr_multiexp}.
    \begin{equation}
        \label{eq:discr_multiexp}
        f(t) = \sum_{i=1}^{n}A_i\exp\left(-\lambda_i t\right) ,
    \end{equation}
    где $n$ -- количество экспоненциальных составляющих в спектре.
    Пример спектра такого сигнала показан на рисунке 
    \ref{pic:multiexp_spect_example}.
    \begin{figure}[h!]
        \centering
        \includegraphics[width=0.5\textwidth]{multiexp_spect_example}
        \caption{Пример спектра сигнала релаксации ёмкости, содержащего
        несколько экспоненциальных составляющих.}
        \label{pic:multiexp_spect_example}
    \end{figure}


    \subsection{Модель аппаратных преобразований спектрометра DLS"~82E}

    В спектрометре DLS-82E реализована корреляционная обработка сигнала
    релаксации ёмкости, таким образом сигнал на выходе аналогового тракта
    спектрометра определяется выражением \ref{eq:dlts_correlation_tech},
    согласно публикации \cite{istratov_exp_analysis}.
    \begin{equation}
        \label{eq:dlts_correlation_tech}
        S\left[g(\lambda),t_c,t_d\right]=\frac{1}{t_c}\int_{t_d}^{t_d+t_c}
        f(t)W\left(t-t_d\right)dt ,
    \end{equation}
    где
    \begin{description}
        \item[$W(t)$] -- весовая функция, определённая на интервале 
        времени $\left[0,t_c\right]$,
        \item[$t_c$] -- период (длительность) весовой функции $W(t)$,
        \item[$t_d$] -- время задержки между началом сигнала релаксации
        и началом корреляционной обработки. Согласно обзору 
        \cite{istratov_exp_analysis}, время задержки $t_d$, обычно, 
        вводится для улучшения избирателности или для снижения искажения
        сигнала из-за перегрузки измерительной системы.
        \item[$g(\lambda)$] -- распределение скоростей экспоненциальных
        спадов, составляющих релаксационный сигнал.
    \end{description}

    Модель аппаратных преобразований (корреляционной обработки),
    учитывающая форму весовой функции, реализованной в спектрометре
    DLS"~82E, для моноэкспоненциального сигнала определяется выражением
    \ref{eq:dls82e_model_S} \cite{rp_vak}.
    \begin{equation}
        \label{eq:dls82e_model_S}
        S\left(\tau,C_A,F_0, t_1\right) = C_A K_{BS} K_{LS} 
        \phi\left(\tau,F_0,t_1\right),
    \end{equation}
    где
    \begin{description}
        \item[$C_A$] -- амплитуда емкостного релаксационного сигнала,
        \item[$K_{BS}$] -- масштабный коэффициент, зависящий от 
        чувствительности емкостного моста,
        \item[$K_{LS}$] -- масштабный коэффициент селектора,
        \item[$\tau$] -- постоянная времени релаксации глубокого уровня,
        \item[$F_0$] -- частота сканирования импульсов заполнения,
        \item[$t_1$] -- длительность импульса заполнения,
        \item[$\phi\left(\tau,F_0,t_1\right)$] -- функция определяемая
        выражением \ref{eq:dls82e_model_phi}.
    \end{description}
    \begin{equation}
        \label{eq:dls82e_model_phi}
        \phi\left(\tau,F_0,t_1\right) = 
        M \tau F_0 e^{-\frac{0.05}{\tau F_0}}
        \left(1-e^{\frac{t_1 F_0-0.45}{\tau F_0}}
        -e^{-\frac{0.5}{\tau F_0}}+
        e^{\frac{t_1 F_0-0.95}{\tau F_0}}\right),
    \end{equation}
    где $M$ -- масштабный множитель.

    Масштабный множитель $M$ определяется выражением
    \ref{eq:dls82e_model_M}.
    \begin{equation}
        \label{eq:dls82e_model_M}
        M(\tau, F_0, t_1) = \frac{1}{\max{\left[
        \tau F_0 e^{-\frac{0.05}{\tau F_0}}
        \left(1-e^{\frac{t_1 F_0-0.45}{\tau F_0}}
        -e^{-\frac{0.5}{\tau F_0}}+
        e^{\frac{t_1 F_0-0.95}{\tau F_0}}\right)
        \right]}}
    \end{equation}

    Введём коэффициент $A$ (выражение \ref{eq:dls82e_model_A}), 
    характеризующий амплитуду сигнала релаксации ёмкости и перепишем 
    выражение \ref{eq:dls82e_model_S} с учётом того, что длительность
    импульса заполнения $t_1$ является неизменной величиной, и получим
    выражение \ref{eq:dls82e_model_S_short}.
    \begin{equation}
        \label{eq:dls82e_model_A}
        A = C_A K_{BS} K_{LS}.
    \end{equation}
    \begin{equation}
        \label{eq:dls82e_model_S_short}
        S(\tau,A,F_0) = A\phi(\tau, F_0)
    \end{equation}

    Для одновременного учёта нелинейности аппаратного тракта и 
    неэкспоненциальности сигнала релаксации, связанной с присутствием 
    нескольких экспоненциальных составляющих в модель вводят коэффициент
    нелинейности"=неэкспоненциальности $p$ \cite{rp_vak}, после чего 
    выражение \ref{eq:dls82e_model_S_short} приобретает вид выражения 
    \ref{eq:dls82e_model_S_p}.
    \begin{equation}
        \label{eq:dls82e_model_S_p}
        S(\tau,A,F_0,p) = A\left[\phi(\tau, F_0)\right]^p.
    \end{equation}

    Для моноэкспоненциальных сигналов релаксации коэффициент $p=1$, но,
    как будет показано далее, в случае наличия нескольких экспоенециальных
    составляющих в сигнале релаксации коэффициент $p$ становится меньше~1.


    \subsection{Модель для расчёта исходных данных}
    Если предположить, что сигнал релаксации ёмкости состоит из нескольких
    экспоненциальных составляющих и определяется выражением
    \ref{eq:discr_multiexp}, то опираясь на выражения
    \ref{eq:discr_multiexp}, \ref{eq:dlts_correlation_tech}, 
    \ref{eq:dls82e_model_S_short} и \ref{eq:dls82e_model_phi}, можно
    сделать выод, что частотный скан, созданный таким сигналом релаксации
    ёмкости определяется выражением~\ref{eq:multiexp_frequncy_scan}.
    \begin{equation}
        \label{eq:multiexp_frequncy_scan}
        Y = \sum_{i=1}^{n} A_i \phi(\tau_i, F_0) ,
    \end{equation}
    где $n$ -- количество экспоненциальных составляющих в сигнале 
    релаксации.

    \section{Моделирование и его результаты}
	В данном разделе приводится краткое описание программной реализации
	модели, примеры рассчитанных частотных сканов, примеры результатов
	идентификации их моделей и график зависимости коэффициента 
	нелинейности"=неэкспоненциальности $p$ от расстояния между крайними
	линиями на спектре.

	\subsection{Реализация модели}
	Модель (выражение \ref{eq:dls82e_model_S_p}) реализована на 
	языке программирования Python (версия 3.9.12) с применением 
	библиотеки TensorFlow (версия 2.8.0) и других библиотек	для научных
	вычислений.

	Модель частотного скана реализованна в виде отдельного класса \\
	\emph{FrequencyScan()} в модуле \emph{fsmodels.py}. Код 
	прокомментирован. Все параметры снабжены адекватными значениями по 
	умолчанию. В дальнейшем планируется дополнение документации и 
	перенос всех программных инструментов для обработки 
	экспериментальных данных в один пакет.

	Модель реализует две функции:
	\begin{enumerate}
		\item Вычисление частотного скана по заданным параметрам и 
		заданному вектору десятичных логарифмов частот опорной функции.
		\item Идентификация параметров модели частотного скана по 
		экспериментальным данным.
	\end{enumerate}
	Имеется возможность вывода значений параметров модели на каждой 
	итерации при идентификации. Примеры использования модели можно найти
	в файле \emph{tensorflow\_model.ipynb} (ПО Jupyter Notebook в составе 
	дистрибутива Anaconda).

	Программа при каждом вычислении значения $\phi\left(\tau,F_0,
	t_1\right)$	(выражение~\ref{eq:dls82e_model_phi}) находит 
	\(
		\max{\left[
	    \tau F_0 e^{-\frac{0.05}{\tau F_0}}
	    \left(1-e^{\frac{t_1 F_0-0.45}{\tau F_0}}
	    -e^{-\frac{0.5}{\tau F_0}}+
	    e^{\frac{t_1 F_0-0.95}{\tau F_0}}\right)
	    \right]}
    \)
	методом	градиентного спуска и вычисляет масштабный множитель $M$ 
	(выражение \ref{eq:dls82e_model_M}).

	Идентификация параметров модели производится методом градиетного 
	спуска, при этом минимизируется среднеквадратическая ошибка между 
	значениями, полученными в результате измерений, и результатами 
	моделирования (выражение \ref{eq:mse}).
	\begin{equation}
		\label{eq:mse}
		E = \frac{1}{n}\sum_{i=1}^{n}\left(y_i - y_i^*\right)^2,
	\end{equation}
	где
	\begin{description}
		\item[$y_i$] -- значения, полученные в результате измерений,
		\item[$y_i^*$] -- значения, полученные в результате моделирования,
		\item[$n$] -- количество измерений.
	\end{description}

	Градиентный спуск везде реализован с помощью библиотеки TensorFlow,
	которая использует алгоритм дифференцирования на графе вычислений,
	таким образом, производная берётся символьно (точно), затем вычисляется
	её значение, поэтому точность вычисления градиента ограничена только
	разрядностью чисел \cite{hands_on_ml}.

	\textbf{Для ускорения процесса идентификации и улучшения сходимости
	в модели вместо постоянной времени сигнала релаксации $\tau$ 
	выполняется идентификация величины $\rho = \log_{10}(\tau)$. По этим же
	и некоторым	другим техническим причинам при вычислении частотного 
	скана на вход модели нужно подавать не вектор частот опорной 
	функции, а вектор их десятичных логарифмов.}

	На рисунке \ref{pic:identification_test} показан пример результата
	идентификации модели на тестовых (специально сгенерированных) данных.

	\begin{figure}[h!]
		\centering
		\includegraphics[width=0.75\textwidth]{identification_test}
		\caption{Пример результата идентификации модели.}
		\label{pic:identification_test}
	\end{figure}

	На рисунке \ref{pic:param_path} показан <<путь>> изменеия параметров 
	(десятичного логарифма постоянной времени и амплитуды) при
	идентификации. Красными точками отмечены значения параметров на 
	каждой итерации, изолинии показывают значения среднеквадратической
	ошибки. 

	В идеальном случае изолинии должны иметь форму концентрических
	окружностей, а <<траекторя>> значений параметров должна быть прямой
	линией (в случае обычного градиентного спуска), направленной к центру
	данных окружностей. Таким образом, рисунок \ref{pic:param_path} 
	позволяет сделать вывод о возможности повышения скорости идентификации.

	\begin{figure}[h!]
		\centering
		\includegraphics[width=0.75\textwidth]{path}
		\caption{<<Путь>> значений параметров при идентификации.}
		\label{pic:param_path}
	\end{figure}


	\subsection{Результаты моделирования}

	Для определения формы зависимости показателя 
	нелинейности"=неэкспоненциальности $p$ было вы полнено моделирование 
	31 частотного скана со спектрами, состоящими из трёх экспоненциальных
	составляющих единичной амплитуды. При моделировании предполагалось,
	что условия идеальны и частотные сканы не содержат шума.

	На первом этапе моделирования, соглассно выражению \ref{eq:multiexp_frequncy_scan},
	рассчитывался частотный скан, затем производилась идентификация 
	параметров модели этого частотного скана, согласно выражению 
	\ref{eq:dls82e_model_S_p}. Идентификация проводилась по всем точкам.

	Результаты моделирования некоторых частотных сканов и спектры, по 
	которым эти частотные сканы были рассчитаны, приведены на рисунках~
	\ref{pic:result_0},	\ref{pic:result_1}, \ref{pic:result_2}. 
	В таблице~\ref{table:results}, вынесенной в приложение приведены 
	полные результаты моделирования. На рисунке~\ref{pic:p_delta_tau} 
	показан график полученной зависимости показателя	
	нелинейности"=неэспоненциальности от разности постоянных времени
	крайних на спектре линий.

	\begin{figure}[h!]
		\centering
		\begin{subfigure}[c]{0.3\textwidth}
			\includegraphics[width=\linewidth]{spectr0}
			\caption{Спектр}
			\label{pic:results_0_spectr}
		\end{subfigure}
		\begin{subfigure}[c]{0.4\textwidth}
			\includegraphics[width=\linewidth]{identification_results_0}
			\caption{Рассчитанный частотный скан и результаты идентификации}
			\label{pic:results_0_scan}
		\end{subfigure}
		\caption{Результаты моделирования частотного скана №1}
		\label{pic:result_0}
	\end{figure}

	\begin{figure}[h!]
		\centering
		\begin{subfigure}[c]{0.3\textwidth}
			\includegraphics[width=\linewidth]{spectr17}
			\caption{Спектр}
			\label{pic:results_1_spectr}
		\end{subfigure}
		\begin{subfigure}[c]{0.4\textwidth}
			\includegraphics[width=\linewidth]{identification_results_17}
			\caption{Рассчитанный частотный скан и результаты идентификации}
			\label{pic:results_1_scan}
		\end{subfigure}
		\caption{Результаты моделирования частотного скана №18}
		\label{pic:result_1}
	\end{figure}

	\begin{figure}[h!]
		\centering
		\begin{subfigure}[c]{0.3\textwidth}
			\includegraphics[width=\linewidth]{spectr30}
			\caption{Спектр}
			\label{pic:results_2_spectr}
		\end{subfigure}
		\begin{subfigure}[c]{0.45\textwidth}
			\includegraphics[width=\linewidth]{identification_results_30}
			\caption{Рассчитанный частотный скан и результаты идентификации}
			\label{pic:results_2_scan}
		\end{subfigure}
		\caption{Результаты моделирования частотного скана №31}
		\label{pic:result_2}
	\end{figure}

	\begin{figure}[h!]
		\centering
		\includegraphics[width=0.7\textwidth]{semilogx_p_func}
		\caption{Зависимость показателя $p$ от разности постоянных времени
		крайних на спектре линий.}
		\label{pic:p_delta_tau}
	\end{figure}

    \section{Выводы}
	\begin{enumerate}
		\item Полученные результаты моделирования частотных сканов 
		(рисунок~\ref{pic:p_delta_tau}, таблица~\ref{table:results})
		позволяют сделать вывод о том, что коэффициент 
		нелинейности"=неэкспоненциальности $p$ нелинейно убывает с 
		увеличением разности постоянных времени крайних линий на спектре
		($\tau_0 - \tau_2$).
		\item С увеличение разности постоянных времени крайних линий на 
		спектре растёт среднеквадратическая ошибка между результатами,
		полученными на идентифицированной модели, и исходными данными.
		\item Возможно улучшение программной реализации модели и повышение
		скорости идентификации её параметров.
	\end{enumerate}

    \pagebreak
    
    \printbibliography[heading=bibintoc]

    \pagebreak

    \section*{Приложение 1}
\addcontentsline{toc}{section}{Приложение 1}
	\begin{table}[ht]
	\begin{tabular}{|l|l|l|l|l|l|l|l|l|l|l|}
        \hline
		№  & $A$  & $\log10(\tau)$ & $p$   & $E$      & $\tau_0$ & $\tau_1$ & $\tau_2$ & $A_0, A_1, A_2$\\ \hline
		0  & 3    & -2             & 1     & 7.65E-10 & 0.01     & 0.01     & 0.01     & 1              \\ \hline
		1  & 3    & -2             & 0.999 & 3.04E-08 & 0.0104   & 0.01     & 0.00962  & 1              \\ \hline
		2  & 3    & -2             & 0.997 & 4.73E-07 & 0.0108   & 0.01     & 0.00926  & 1              \\ \hline
		3  & 2.99 & -2             & 0.994 & 2.38E-06 & 0.0112   & 0.01     & 0.00891  & 1              \\ \hline
		4  & 2.98 & -2             & 0.989 & 7.47E-06 & 0.0117   & 0.01     & 0.00858  & 1              \\ \hline
		5  & 2.97 & -2             & 0.984 & 1.81E-05 & 0.0121   & 0.01     & 0.00825  & 1              \\ \hline
		6  & 2.96 & -2             & 0.976 & 3.71E-05 & 0.0126   & 0.01     & 0.00794  & 1              \\ \hline
		7  & 2.95 & -2             & 0.968 & 6.78E-05 & 0.0131   & 0.01     & 0.00764  & 1              \\ \hline
		8  & 2.93 & -2             & 0.959 & 0.000114 & 0.0136   & 0.01     & 0.00736  & 1              \\ \hline
		9  & 2.92 & -2             & 0.948 & 0.00018  & 0.0141   & 0.01     & 0.00708  & 1              \\ \hline
		10 & 2.9  & -2             & 0.937 & 0.000269 & 0.0147   & 0.01     & 0.00681  & 1              \\ \hline
		11 & 2.88 & -2             & 0.924 & 0.000386 & 0.0153   & 0.01     & 0.00656  & 1              \\ \hline
		12 & 2.85 & -2             & 0.911 & 0.000535 & 0.0158   & 0.01     & 0.00631  & 1              \\ \hline
		13 & 2.83 & -2             & 0.897 & 0.000719 & 0.0165   & 0.01     & 0.00607  & 1              \\ \hline
		14 & 2.81 & -2             & 0.882 & 0.000943 & 0.0171   & 0.01     & 0.00584  & 1              \\ \hline
		15 & 2.78 & -2             & 0.866 & 0.00121  & 0.0178   & 0.01     & 0.00562  & 1              \\ \hline
		16 & 2.75 & -2             & 0.85  & 0.00152  & 0.0185   & 0.01     & 0.00541  & 1              \\ \hline
		17 & 2.72 & -2             & 0.834 & 0.00188  & 0.0192   & 0.01     & 0.00521  & 1              \\ \hline
		18 & 2.69 & -2             & 0.817 & 0.00229  & 0.02     & 0.01     & 0.00501  & 1              \\ \hline
		19 & 2.66 & -2             & 0.799 & 0.00274  & 0.0207   & 0.01     & 0.00482  & 1              \\ \hline
		20 & 2.63 & -2             & 0.781 & 0.00325  & 0.0215   & 0.01     & 0.00464  & 1              \\ \hline
		21 & 2.6  & -2             & 0.763 & 0.00381  & 0.0224   & 0.01     & 0.00447  & 1              \\ \hline
		22 & 2.57 & -2             & 0.745 & 0.00441  & 0.0233   & 0.01     & 0.0043   & 1              \\ \hline
		23 & 2.53 & -1.99          & 0.727 & 0.00506  & 0.0242   & 0.01     & 0.00414  & 1              \\ \hline
		24 & 2.5  & -1.99          & 0.708 & 0.00574  & 0.0251   & 0.01     & 0.00398  & 1              \\ \hline
		25 & 2.47 & -1.99          & 0.69  & 0.00647  & 0.0261   & 0.01     & 0.00383  & 1              \\ \hline
		26 & 2.43 & -1.99          & 0.671 & 0.00723  & 0.0271   & 0.01     & 0.00369  & 1              \\ \hline
		27 & 2.4  & -1.99          & 0.653 & 0.00803  & 0.0282   & 0.01     & 0.00355  & 1              \\ \hline
		28 & 2.36 & -1.99          & 0.635 & 0.00884  & 0.0293   & 0.01     & 0.00341  & 1              \\ \hline
		29 & 2.33 & -1.99          & 0.616 & 0.00968  & 0.0304   & 0.01     & 0.00329  & 1              \\ \hline
		30 & 2.29 & -1.98          & 0.598 & 0.0105   & 0.0316   & 0.01     & 0.00316  & 1              \\ \hline
	\end{tabular}
\end{table}
    В таблице \ref{table:results} использованы следующие обозначения:
    \begin{description}
    	\item[$\tau_0, \tau_1, \tau_2$] -- Заданные постоянные времени
    	экспоненциальных составляющих на спектре.
    	\item[$A_0, A_1, A_2$] -- Заданные амплитуды экспоненциальных составляющих на спектре. В данном случае амплитуды всех
    	составляющих равны 1.
    	\item[$p$] -- коэффициент нелинейности"=неэкспоненциальности,
    	полученный в результате идентификации параметров модели.
    	\item[$\tau$] -- постоянная времени, полученная в результате
    	идентификации параметров модели.
    	\item[$A$] -- амплитуда, полученная в результате идентификации
    	параметров модели.
    	\item[$E$] -- среднеквадратическая ошибка.
    \end{description}

\end{document}